\documentclass{article}

\usepackage{martin-style}

\begin{document}

	\begin{proposition}
	Puzzle 1: Find an infinite collection of open intervals in $\mathbb{R}$ whose intersection is not an open interval.
	\end{proposition}
	\begin{proof}
	$\bigcap\lbrace (\frac{-1}{n}, \frac{1}{n}) : n \in \mathbb{Z}^+\rbrace = \lbrace 0 \rbrace$ which is not an open interval or open.
	\end{proof}

  \begin{proposition}
      Any finite union of closed sets is closed, and any arbitrary intersection of closed sets is closed.
  \end{proposition}
  \begin{proof}
  	Proposition 1 - We proceed by showing that any finite union of closed sets is closed:
  	\newline
  	Let C, D be closed sets.
  	\newline
  	Let A, B be compliments of C, D so A, B are open.
  	\newline
  	Then, A $\cap$ B is also open.
  	\newline
  	Thus, X$\setminus$(A $\cap$ B) is closed.
  	\newline
  	By Demorgan's Law, X$\setminus$(A $\cap$ B) = X$\setminus$A $\cup$ X$\setminus$B = C $\cup$ D which is closed.
  	\newline\newline
	Now that we've shown that C $\cup$ D is closed for all C, D; Assume C$_1$ $\cup $ C$_2 \cup ...$ $\cup $ C$_n$ is closed for closed sets C$_i$. Then for C$_1$ $\cup$ ... $\cup$ C$_n$ $\cup$ C$_{n+1}$, let K = C$_1$ $\cup$ ... $\cup$ C$_n$. Thus K $\cup$ C$_{n+1}$ = C$_1$ $\cup$ ... $\cup$ C$_n$ $\cup$ C$_{n+1}$ is closed.
	\newline
	Now we show that any arbitrary intersection of closed sets is closed.
  	\newline
  	Let $\mathcal{C}$ be a collection of closed sets.
  	\newline
  	Let $\mathcal{U} = \lbrace$X$\setminus$C : C $\in \mathcal{C}\rbrace$, so $\mathcal{U}$ is a collection of open sets.
  	\newline
  	Then, $\bigcup \mathcal{U}$ is also open.
  	\newline
  	Thus, X$\setminus\bigcup\mathcal{U}$ is closed.
  	\newline
  	By Demorgan's Law, X$\setminus\bigcup\mathcal{U}$ = $\bigcap\mathcal{C}$
  	\newline
  	Therefore, $\bigcap\mathcal{C}$ is closed.
	\end{proof}

	\begin{lemma}
	A set U is open if and only if for every point $x \in$ U, there exists an open set U$_x$ where $x \in $ U$_x \subseteq$ U
	\end{lemma}
	\begin{proof} Suppose U is open, then for all x $\in$ U there exists an open U$_x$ = U, such that x $\in$ U$_x \subseteq$ U.
	\newline
	To show the converse, for each x $\in$ U there is an open set U$_x$ where x $\in$ U$_x$ $\subseteq$ U. To show that U = $\bigcup \lbrace$ U$_x$ : x $\in$ U $\rbrace$, let x $\in$ U, then for each U$_x$, U $\subseteq$ U$_x$. Thus U $\subseteq$ $\bigcup \lbrace$ U$_x$ : x $\in$ U $\rbrace$. Now let y $\in$ U$_x$ for some x $\in$ U, then U$_x$ $\subseteq$ U. Thus U $\supseteq$ $\bigcup \lbrace$ U$_x$ : x $\in$ U $\rbrace$. Therefore U = $\bigcup \lbrace$ U$_x$ : x $\in$ U $\rbrace$.
	\end{proof}

	\begin{proposition}
		A set K in a topological space X is closed if and only if K contains all its limit points.
	\end{proposition}
	\begin{proof} Suppose K contains all its limit points.
	If K = X, then K is closed because $\emptyset$ is open.	Otherwise let x $\in$ X$\setminus$K, so x is not a limit point of K.	Then, $\exists$ $x \in$ U$_x \in \tau$ such that U$_x \cap$ K = $\emptyset$.	So, $x \in $ U$_x \subseteq$ X$\setminus$K.	By the lemma, X$\setminus$K is open so K is closed.

	To show the converse is true as well, let x $\in$ X$\setminus$K, which is an open set by the lemma. Since $($X$\setminus$K$)\cap$K is the empty set, x is not a limit point of K. So, if $\ell$ is any limit point of K, then $\ell \not\in$ X$\setminus$K, so $\ell \in$ K and K contains all of its limit points.
	\end{proof}

	\begin{proposition}
	Verify the discrete and indiscrete topologies are topologies
	\end{proposition}
	\begin{proof} To prove the discrete topology is an actual topology, first we define that the discrete topology on a set X is $\tau$ = $\mathcal{P}($X$)$.	$\emptyset$, X $\in \mathcal{P}($X$)$, So $\emptyset,$ X $\in \tau$. Now, let $\mathcal{U} \subseteq \tau$ = $\mathcal{P}($X$)$, then the $\bigcup$U $\in \mathcal{P}($X$)$. Let U, V $\in \tau$ = $\mathcal{P}($X$)$ then the intersection U $\cap$ V $\in \mathcal{P}($X$)$ = $\tau$. Now to show that the indiscrete topology is an actual topology, we define that the indiscrete topology on a set X is given as $\tau$ = $\lbrace \emptyset$, X$\rbrace$. Clearly, $\emptyset$, X $\in \tau$. Let $\mathcal{U}$ be a collection of open sets in X, then $\bigcup \mathcal{U}$ = X, thus $\bigcup \mathcal{U} \in \tau$.
	\end{proof}

	\begin{definition}
		A collection of sets \(\mathcal{B}\subseteq\mathcal{P}(X)\) is
		called a basis if:
		\begin{enumerate}
			\item For all \(x\in X\), there exists \(B\in\mathcal{B}\) such that
			\(x\in B\).
			\item  For all \(B_1,B_2 \in \mathcal{B}\) with \(x \in B_1 \cap B_2\), there exists \(B_3 \in \mathcal{B}\) with
			\(x\in B_3 \subseteq B_1 \cap B_2\).
		\end{enumerate}
	  The set \(\{\bigcup\mathcal{B}':\mathcal{B}'\subseteq\mathcal{B}\}\)
		is called the topology generated by \(\mathcal{B}\).
	\end{definition}

	\begin{theorem}
		The ``topology generated by \(\mathcal{B}\)'' is actually a topology.
	\end{theorem}

	\begin{theorem}
	Let $\tau$ be a topology on X. Then $\mathcal{B} \subseteq \tau$ generates $\tau$ if:
	\newline
	1. For all x $\in$ U $\in \tau$, there exists B $\in$ $\mathcal{B}$ where x $\in$ B $\subseteq$ U $\in \tau$
	\newline
	2. For all B$_1$, B$_2 \in \mathcal{B}$ with x $\in ($B$_1 \cap$ B$_2)$, there exists B$_3 \in \mathcal{B}$ with x $\in$ B$_3 \subseteq$ B$_1 \cap$ B$_2$.
\end{theorem}

	\begin{theorem}
	If $\mathcal{B}$ is a basis for $\tau$, then $\tau$ = $\lbrace \bigcup \mathcal{B}^{'}$ : $\mathcal{B}^{'} \in \mathcal{B} \rbrace$. (Every open set is a union of basic sets)
	\end{theorem}
	\begin{proof}
	Let $\mathcal{B} \subseteq \tau$, then for any $\mathcal{B}^{'} \in \mathcal{B}$, $\mathcal{B}^{'} \in \tau$. Furthermore the $\bigcup \mathcal{B}^{'}$ = $\mathcal{B}$, then $\bigcup \mathcal{B}^{'} \subseteq \tau$ and $\bigcup \mathcal{B}^{'}$ = $\tau$.
	\end{proof}

	\begin{theorem}
	$\lbrace\mathbb{X}\rbrace$ is a basis for $\tau$ = $\lbrace\emptyset$, $\mathbb{X}\rbrace$. (indiscrete)
	\end{theorem}
	\begin{proof}
	$\mathbb{X}$ $\in \lbrace \emptyset$, $\mathbb{X}\rbrace$ = $\tau$. First, let U = $\mathbb{X} \in \tau$, and let x $\in$ U $\in \tau$, then for all x $\in$ U, x $\in \mathbb{X}$, and for some X$_n \in \mathbb{X}$, x $\in$ X$_n$. Thus, x $\in$ X$_n \subseteq$ U $\in \tau$. Now, we let x $\in$ X$_1$, X$_2$. Then x $\in ($X$_1 \cap $X$_2)$. Let X$_3 \subseteq ($X$_1 \cap $X$_2)$ such that X$_3$ = $($X$_1 \cap$ X$_2)$ then x $\in$ X$_3$. Thus, x $\in$ X$_3 \subseteq ($X$_1 \cap $X$_2) \in \mathbb{X}$. Then $\lbrace \mathbb{X} \rbrace$ is a basis for $\tau$ = $\lbrace \emptyset$, $\mathbb{X} \rbrace$.
	\end{proof}

	\begin{theorem}
	$\lbrace\lbrace$X$\rbrace$ : x $\in$ X$\rbrace$ is a basis for $\tau$ = $\mathcal{P}($X$)$ = $\lbrace$ U : U $\subseteq$ X$\rbrace$. (discrete)
	\end{theorem}
	\begin{proof}
	The $\bigcup \lbrace$ U : U $\subseteq \mathbb{X} \rbrace$ = $\mathbb{X}$, so for all x $\in$ U, x $\in \mathbb{X} \in \tau$. So, there exists X $\subseteq \mathbb{X}$ such that x $\in$ X, thus x $\in$ X $\subseteq \mathbb{X} \in \tau$. Again, the $\bigcup \lbrace$ U : U $\subseteq \mathbb{X} \rbrace$ = $\mathbb{X}$, so let x $\in$ X$_1$, X$_2 \subseteq \mathbb{X}$ then x $\in \mathbb{X}$, and x $\in ($X$_1 \cap$ X$_2)$ = X$_3$. X$_3$ = $($X$_1 \cap$ X$_2) \subseteq \mathbb{X}$, so X$_3 \in \mathbb{X}$ with x $\in$ X$_3 \subseteq ($X$_1 \cap$ X$_2) \subseteq \mathbb{X} \in \tau$.
	\end{proof}

	\begin{definition}
	The Euclidean topology on $\mathbb{R}$ is the topology generated by the basis $\lbrace($a, b$)$ : a $<$ b are in $\mathbb{R}\rbrace$.
	\end{definition}

	\begin{theorem}
	$\lbrace($a, b$)$ : a $<$ b $\in \mathbb{R}\rbrace$ is a basis.
	\end{theorem}
	\begin{proof}
	\end{proof}

	\begin{theorem}
	$\lbrace($a, b$)$ : a $<$ b $\in \mathbb{Q}\rbrace$ is a basis for the Euclidean topology.
	\end{theorem}
	\begin{proof}
	\end{proof}

\end{document}
