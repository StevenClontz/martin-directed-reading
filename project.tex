\documentclass{article}

\usepackage{martin-style}

\begin{document}

	\begin{proposition}
	Find an infinite collection of open intervals in $\mathbb{R}$ whose intersection is not an open interval.
	\end{proposition}
	\begin{proof}
	$\bigcap\lbrace (\frac{-1}{n}, \frac{1}{n}) : n \in \mathbb{Z}^+\rbrace = \lbrace 0 \rbrace$ which is not an open interval or even open.
	\end{proof}

  \begin{proposition}
      Any finite union of closed sets is closed, and any arbitrary intersection of closed sets is closed.
  \end{proposition}
  \begin{proof}
	We proceed by showing that any finite union of closed sets is closed:
  	\newline
  	Let C, D be closed sets.
  	\newline
  	Let A, B be compliments of C, D so A, B are open.
  	\newline
  	Then, A $\cap$ B is also open.
  	\newline
  	Thus, X$\setminus$(A $\cap$ B) is closed.
  	\newline
  	By Demorgan's Law, X$\setminus$(A $\cap$ B) = X$\setminus$A $\cup$ X$\setminus$B = C $\cup$ D which is closed.
  	\newline\newline
	Now that we've shown that C $\cup$ D is closed for all C, D; Assume C$_1$ $\cup $ C$_2 \cup ...$ $\cup $ C$_n$ is closed for closed sets C$_i$. Then for C$_1$ $\cup$ ... $\cup$ C$_n$ $\cup$ C$_{n+1}$, let K = C$_1$ $\cup$ ... $\cup$ C$_n$.\newline Thus K $\cup$ C$_{n+1}$ = C$_1$ $\cup$ ... $\cup$ C$_n$ $\cup$ C$_{n+1}$ is closed.
	\newline
	Now we show that any arbitrary intersection of closed sets is closed.
  	\newline
  	Let $\mathcal{C}$ be a collection of closed sets.
  	\newline
  	Let $\mathcal{U} = \lbrace$X$\setminus$C : C $\in \mathcal{C}\rbrace$, so $\mathcal{U}$ is a collection of open sets.
  	\newline
  	Then, $\bigcup \mathcal{U}$ is also open.
  	\newline
  	Thus, X$\setminus\bigcup\mathcal{U}$ is closed.
  	\newline
  	By Demorgan's Law, X$\setminus\bigcup\mathcal{U}$ = $\bigcap\mathcal{C}$
  	\newline
  	Therefore, $\bigcap\mathcal{C}$ is closed.
	\end{proof}

	\begin{lemma}
	A set U is open if and only if for every point $x \in$ U, there exists an open set U$_x$ where $x \in $ U$_x \subseteq$ U
	\end{lemma}
	\begin{proof} Suppose U is open, then for all x $\in$ U there exists an open U$_x$ = U, such that x $\in$ U$_x \subseteq$ U.
	\newline
	To show the converse, suppose that for each x $\in$ U there is an open set U$_x$ where x $\in$ U$_x$ $\subseteq$ U. For $x \in U, x \in U_x$ so $x \in \bigcup \lbrace U_x : x \in U \rbrace$. Thus $U \subseteq \bigcup \lbrace U_x : x \in U_x \rbrace$. Now let y $\in$ U$_x$ for some x $\in$ U, then U$_x$ $\subseteq$ U. Thus U $\supseteq$ $\bigcup \lbrace$ U$_x$ : x $\in$ U $\rbrace$. Therefore U = $\bigcup \lbrace$ U$_x$ : x $\in$ U $\rbrace$.
	\end{proof}

	\begin{proposition}
		A set K in a topological space X is closed if and only if K contains all its limit points.
	\end{proposition}
	\begin{proof} Suppose K contains all its limit points.
	If K = X, then K is closed because $\emptyset$ is open.	Otherwise let x $\in$ X$\setminus$K, so x is not a limit point of K.	Then, $\exists$ $x \in$ U$_x \in \tau$ such that U$_x \cap$ K = $\emptyset$ since $x \not\in K$.	So, $x \in $ U$_x \subseteq$ X$\setminus$K.	By the lemma, X$\setminus$K is open so K is closed.

	To show the converse is true as well, let x $\in$ X$\setminus$K, which is an open set by the lemma. Since $($X$\setminus$K$)\cap$K is the empty set, x is not a limit point of K. So, if $\ell$ is any limit point of K, then $\ell \not\in$ X$\setminus$K, so $\ell \in$ K and K contains all of its limit points.
	\end{proof}

	\begin{proposition}
	Verify the discrete and indiscrete topologies are topologies
	\end{proposition}
	\begin{proof} To prove the discrete topology is an actual topology, first we define that the discrete topology on a set X is $\tau$ = $\mathcal{P}($X$)$.	$\emptyset$, X $\in \mathcal{P}($X$)$, So $\emptyset,$ X $\in \tau$. Now, let $\mathcal{U} \subseteq \tau$ = $\mathcal{P}($X$)$, then the $\bigcup$U $\in \mathcal{P}($X$)$. Let U, V $\in \tau$ = $\mathcal{P}($X$)$ then the intersection U $\cap$ V $\in \mathcal{P}($X$)$ = $\tau$. Now to show that the indiscrete topology is an actual topology, we define that the indiscrete topology on a set X is given as $\tau$ = $\lbrace \emptyset$, X$\rbrace$. Clearly, $\emptyset$, X $\in \tau$. Let $\mathcal{U}$ be a collection of open sets in X, then $\bigcup \mathcal{U}$ = X, thus $\bigcup \mathcal{U} \in \tau$.
	\end{proof}

	\begin{definition}
		A collection of sets \(\mathcal{B}\subseteq\mathcal{P}(X)\) is
		called a basis if:
		\begin{enumerate}
			\item For all \(x\in X\), there exists \(B\in\mathcal{B}\) such that
			\(x\in B\).
			\item  For all \(B_1,B_2 \in \mathcal{B}\) with \(x \in B_1 \cap B_2\), there exists \(B_3 \in \mathcal{B}\) with
			\(x\in B_3 \subseteq B_1 \cap B_2\).
		\end{enumerate}
	  The set \(\{\bigcup\mathcal{B}':\mathcal{B}'\subseteq\mathcal{B}\}\)
		is called the topology generated by \(\mathcal{B}\).
	\end{definition}

	\begin{theorem}
		The ``topology generated by \(\mathcal{B}\)'' is actually a topology.
	\end{theorem}
	\begin{proof} NEEDS WORK\newline
	$\tau$ is $\lbrace \bigcup \mathcal{B}' : \mathcal{B}' \subseteq \mathcal{B} \rbrace$. So $\lbrace \bigcup \mathcal{B}' : \mathcal{B}' \subseteq \mathcal{B} \rbrace \subseteq \mathcal{P}(X)$. $\emptyset, X \in \tau$ because for $\mathcal{B}' = \emptyset, \bigcup \mathcal{B}' = \emptyset$ and for $\mathcal{B}' = X, \bigcup \mathcal{B}' = X$. Let $\mathcal{U} \subseteq \tau$ and $\bigcup \mathcal{U} \in \tau$ because for each $U \in \mathcal{U}, U = \bigcup \mathcal{B}_{U}'$ for some $\mathcal{B}_{U}' \subseteq \mathcal{B}$ So, $\bigcup \mathcal{U} = \bigcup \lbrace \bigcup \mathcal{B}_{U}' : U \in \mathcal{U} \rbrace$. Let $\mathcal{B}' = \bigcup \lbrace \mathcal{B}_{U}' : U \in \mathcal{U} \rbrace.$ So, $\bigcup \mathcal{U} = \bigcup \mathcal{B}'.$ \newline For $U, V \in \lbrace \bigcup \mathcal{B}' : \mathcal{B}' \subseteq \mathcal{B} \rbrace, U = B_1$ and $V = B_2, B_1, B_2 \in \mathcal{B}$ such that $B_1 = \mathcal{B}$ and $B_2 = \mathcal{B}$. Then $U \cap V \in \lbrace \bigcup \mathcal{B}' : \mathcal{B}' \subseteq \mathcal{B} \rbrace \supseteq W$ so there exists $x \in V, U : x \in U \cap V \supseteq W$. Therefore $x \in W \subseteq U \cap V \in \tau$.
	\end{proof}

	\begin{theorem}
	Let $\tau$ be a topology on X. Then $\mathcal{B} \subseteq \tau$ generates $\tau$ if:
	\newline
	1. For all x $\in$ U $\in \tau$, there exists B $\in$ $\mathcal{B}$ where x $\in$ B $\subseteq$ U $\in \tau$
	\newline
	2. For all B$_1$, B$_2 \in \mathcal{B}$ with x $\in ($B$_1 \cap$ B$_2)$, there exists B$_3 \in \mathcal{B}$ with x $\in$ B$_3 \subseteq$ B$_1 \cap$ B$_2$.
\end{theorem}
\begin{proof}:\newline
	\begin{enumerate}
		\item For $x \in \mathbb{X} \in \tau$, select $x \in B \subseteq \mathbb{X} \in \tau$, so 1 of definition 6 holds.
		\item See definition 6-2.
		\item Let $U \in \tau$, then for every $x \in U$, there exists $B_x \in \mathcal{B}$ where $x \in B_x \subseteq U \in \tau$. Let $\mathcal{B'} = \lbrace B_x : B_x \in \mathcal{B} \rbrace$ so $U = \bigcup \mathcal{B'}$. Now let $\mathcal{B'} \subseteq \mathcal{B} \subseteq \tau$, and since $\mathcal{B'} \subseteq \tau, \bigcup \mathcal{B'} \in \tau$.
		\end{enumerate}
\end{proof}

	\begin{theorem}
	$\lbrace\mathbb{X}\rbrace$ is a basis for $\tau$ = $\lbrace\emptyset$, $\mathbb{X}\rbrace$. (indiscrete)
	\end{theorem}
	\begin{proof}:\newline
	\begin{enumerate}
		\item Let $\mathcal{B} = \lbrace\mathbb{X}\rbrace$ and let $x \in \mathbb{X} = B \in \mathcal{B}$.
		\item Now consider $B_1, B_2 \in \mathcal{B}$. $B_1 = \mathbb{X}$ and $B_2 = \mathbb{X}$ Let $x \in B_1 \cap B_2$, then $x \in B_3 \subseteq B_1 \cap B_2$.
		\item Choose $\mathcal{B}_\emptyset = \emptyset \subseteq \mathcal{B}$ and $\mathcal{B}_\mathbb{X} = \lbrace \mathbb{X} \rbrace \subseteq \mathcal{B}$ where $\bigcup \mathcal{B}_\emptyset = \emptyset$ and $\bigcup \mathcal{B}_\mathbb{X} = \mathbb{X}$, so since $\mathcal{P}(\mathcal{B}) = \lbrace \emptyset, \lbrace \mathbb{X} \rbrace \rbrace$, $\lbrace \bigcup \mathcal{B'} : \mathcal{B'} \subseteq \mathcal{B} \rbrace = \lbrace \bigcup \mathcal{B}_\emptyset, \bigcup \mathcal{B}_\mathbb{X} \rbrace = \tau = \lbrace \emptyset, \lbrace \mathbb{X} \rbrace \rbrace$.
	\end{enumerate}
	\end{proof}

	\begin{theorem}
	$\mathcal{B} = \lbrace\lbrace x \rbrace$ : x $\in \mathbb{X}\rbrace$ is a basis for $\tau$ = $\mathcal{P}(\mathbb{X})$ = $\lbrace$ U : U $\subseteq \mathbb{X} \rbrace$. (discrete)
	\end{theorem}
	\begin{proof}:\newline
		\begin{enumerate}
			\item Let $x \in U \in \tau = \mathcal{P}(\mathbb{X})$. Then for $B = \lbrace x \rbrace \in \mathcal{B}, x \in B \subseteq U$.
			\item Let $B_1, B_2 \in \mathcal{B}$. Let $y \in B_1 \cap B_2$. Let $B_3 = \lbrace y \rbrace$ $y \in B_3 = \lbrace y \rbrace \subseteq B_1 \cap B_2$. 
			\item Let $U \in \tau = \mathcal{P} ( \mathbb{X} )$. Then $U = \lbrace x : x \in U \rbrace = \bigcup \lbrace x : x \in U \rbrace = \bigcup \mathcal{B}_U$ for $\mathcal{B}_U = \lbrace \lbrace x \rbrace : x \in U \rbrace \subseteq \mathcal{B}$.
			\newline Let $\mathcal{B'} \subseteq \mathcal{B}$. Then $\bigcup \mathcal{B'} \in \mathcal{P}(\mathbb{X}) \in \tau$.
		\end{enumerate}
	\end{proof}

	\begin{definition}
	The Euclidean topology on $\mathbb{R}$ is the topology generated by the basis \newline $\lbrace(a, b) : a < b \in \mathbb{R}\rbrace$.
	\end{definition}

	\begin{theorem}
	$\lbrace($a, b$)$ : a $<$ b $\in \mathbb{R}\rbrace$ is a basis.
	\end{theorem}
	\begin{proof}:\newline
		\begin{enumerate}
			\item Let $\mathcal{B} = \lbrace (a, b) : a < b \in \mathbb{R} \rbrace$. Let $x \in \mathbb{R}$ and let $B = (x-1, x+1)$, then $x \in B \subseteq \mathbb{R}$ and $B \in \mathcal{B}$.	
			\item For $B_1 = (a_1, b_1)$, $B_2 = (a_2, b_2)$, $a_1<b_1$, $a_2<b_2 \in \mathbb{R}$ with $x \in B_1 \cap B_2$, then there is $B_3 = ( max(a_1, a_2), min(b_1, b_2) ) \subseteq B_1 \cap B_2$ such that $a_1,a_2<x<b_1,b_2$ so $ x \in B_3$.
		\end{enumerate}
	\end{proof}

	\begin{theorem}
	$\lbrace($a, b$)$ : a $<$ b $\in \mathbb{Q}\rbrace$ is a basis for the Euclidean topology.
	\end{theorem}
	\begin{proof}:\newline
		\begin{enumerate}
		\item Let $\mathcal{B} = \lbrace (a, b) : a < b \in \mathbb{Q}\rbrace$ Let $x \in \mathbb{R}$ then, $\lfloor x \rfloor -1 < x < \lceil x \rceil + 1$. Let B = $(\lfloor x \rfloor -1, \lceil x \rceil + 1)$, so $x \in B \in \mathcal{B}$ and $B \subseteq \mathbb{Q}$.
		\item Now, for $B_1, B_2 \in \mathcal{B}, \exists x \in B_1 \cap B_2 : x \in B_1$ and $x \in B_2$. $a_1, a_2 < b_1, b_2 \in \mathbb{Q}.$ Let $B_1 = (a_1, b_1)$ and $B_2 = (a_2, b_2)$. Now let $B_3 = (max(a_1,a_2), min(b_1,b_2))$ with $a_1,a_2<x<b_1,b_2$ so $x \in B_3$. So, $x \in B_3 \subseteq B_1 \cap B_2$.
		\item Now show it generates $\tau$...
		\end{enumerate} 
	\end{proof}

	\begin{definition}
	  A subset \(K\) of a topological space \(X\) is said to be \textbf{compact} if for every
	  open cover \(\mathcal U\) of \(K\) (every collection \(\mathcal U\) of open sets
	  such that \(\bigcup \mathcal U\supseteq K\)) there exists a finite subcollection
	  \(\mathcal F\subseteq \mathcal U\) that also covers \(K\) (that is,
	  \(\bigcup\mathcal F\supseteq K\)).
	\end{definition}

	\begin{theorem}
	  Any finite union of compact subsets is compact.
	\end{theorem}

	\begin{proof}
	$K_1$ and $K_2$ are compact, so let $\mathcal{U}_3$ be an open cover of $K_3 = K_1 \cup K_2$. So let $\mathcal{U}_1 = \mathcal{U}_3, \bigcup \mathcal{U}_3 \supseteq K_1$ and $\mathcal{U}_2 = \mathcal{U}_3, \bigcup \mathcal{U}_3 \supseteq K_2$. Since $K_1, K_2$ are compact choose finite $\mathcal{F}_1 \subseteq \mathcal{U}_1 = \mathcal{U}_3$ and $\mathcal{F}_2 \subseteq \mathcal{U}_2 = \mathcal{U}_3$, so $\bigcup \mathcal{F}_1 \supseteq K_1$ and $\bigcup \mathcal{F}_2 \supseteq K_2$ and $\bigcup (\mathcal{F}_1 \cup \mathcal{F}_2) \supseteq K_3$. Since $\mathcal{F}_1 \cup \mathcal{F}_2$ is a finite subset of $\mathcal{U}_3, K_3$ is compact.
	\end{proof}

	\begin{proposition}
	  Any finite subset of a topological space is compact.
	\end{proposition}

	\begin{proof}
	For $F = \lbrace x_i : 0 \leq i < n \rbrace \subseteq \mathbb{X}$, let $\mathcal{U}$ be an open covering of $F$ so $\bigcup \mathcal{U} \supseteq F$. For each $0 \leq i < n$, let $U_i \in \mathcal{U}$ satisfy $x_i \in U_i$. Let finite $\mathcal{F} = \lbrace U_i : 0 \leq i < n \rbrace \subseteq \mathcal{U}$, so $\bigcup \mathcal{F} \supseteq F$ and $F$ is compact.
	\end{proof}

	\begin{proposition}
	  The subset \(\{0\}\cup\{\frac{1}{n}:n\in\mathbb Z^+\}\) of \(\mathbb R\) is compact.
	\end{proposition}

	\begin{proof}
	Let $K = \lbrace 0 \rbrace \cup \lbrace \frac{1}{n} : n \in \mathbb{Z^+} \rbrace \subseteq \mathbb{R}$ and let $\mathcal{U}$ be an open covering of $K$. We may choose $U_0 \in \mathcal{U}$ where $0 \in U_0$. So pick $a<b \in \mathbb{R}$ where $0 \in (a, b) \subseteq U_0$. Let $N \in \mathbb{Z}^+$ where $0 < \frac{1}{N} < b$. Now let finite $F = \lbrace \frac{1}{m} : 1 \leq m < N \rbrace$ so it is compact, and choose $\mathcal{F} \subseteq \mathcal{U}$ with $\bigcup \mathcal{F} \supseteq F$. So, $\lbrace \bigcup \mathcal{F} \cup (a,b) \cup U_0 \rbrace \supseteq K$.
	\end{proof}

	\begin{proposition}
	  Any unbounded subset of \(\mathbb R\) is not compact.
	\end{proposition}
	\begin{proof}
	Let $K$ be an unbounded subset of $\mathbb{R}$. Let $\mathcal{U} = \lbrace (n-2, n) : n \in \mathbb{Z} \rbrace$. Let $n_i \in \mathbb{Z}$ be arbitrary for $0 \leq i < n$. Then $\mathcal{F} = \lbrace (n_i - 2, n_i) : 0 \leq i < n \rbrace$ is an arbitrary finite sub collection of $\mathcal{U}$. However, $\bigcup \mathcal{F} \not\supseteq K$, for $min(n_i-2: 0 \leq i < n)$ and $max(n_i : 0 \leq i < n)$ we can find a $k \in K : k < min(n_i-2: 0 \leq i < n)$ or $k > max(n_i : 0 \leq i < n)$, so $k  \not\in \bigcup \mathcal{F}$ and $K$ is not compact.
	\end{proof}	

	\begin{proposition}
	  Any open interval \((a,b)\subseteq\mathbb R\) is not compact.
	\end{proposition}
	\begin{proof}
	Let $K = (a, b) \subseteq \mathbb{R}$, Let $\mathcal{U} = \lbrace (x, y) : x, y \in \mathbb{R}, a < x < y < b \rbrace$. So, $\bigcup \mathcal{U} \supseteq K$, and $\bigcup \mathcal{U} = K$. Let $n \in \mathbb{Z}$, let $x_i, y_i \in \mathbb{R}$ be arbitrary with $0 \leq i < n$. Then, $\mathcal{F} = \lbrace (x_i, y_i) : 0 \leq i < n \rbrace$ is an arbitrary finite subset of $\mathcal{U}$. However $\bigcup \mathcal{F} \not\supseteq K$, for $min(x_i : 0 \leq i < n)$ we can find a $k \in K : k < min(x_i : 0 \leq i < n)$ which is greater than $a$ so $k \not\in \bigcup \mathcal{F}$ and $K$ is not compact.
	\end{proof}

	\begin{definition}
	  A subset \(K\) of a topological space \(X\) is said to be \textbf{Lindel\"of} if for every
	  open cover \(\mathcal U\) of \(K\) (every collection \(\mathcal U\) of open sets
	  such that \(\bigcup \mathcal U\supseteq K\)) there exists a countable subcollection
	  \(\mathcal F\subseteq \mathcal U\) that also covers \(K\) (that is,
	  \(\bigcup\mathcal F\supseteq K\)).
	\end{definition}

	\begin{definition}
	  A subset \(K\) of a topological space \(X\) is said to be \textbf{\(\sigma\)-compact} if
	  there exist compact subspaces \(K_n\) of \(X\) for \(n\in\mathbb N\) such that
	  \(K=\bigcup_{n\in\mathbb N} K_n\).
	  \newline \newline
	\end{definition}
	\begin{theorem}
		Every \(\sigma\)-compact subset of a topological space is Lindel\"of.
	\end{theorem}
	\begin{proof}
	Let $K$ be \(\sigma\)-compact. Let $\mathcal{F}_n = \lbrace U_{n0}, U_{n1}, ..., U_{nm} \rbrace$ be a finite subcover of $\mathcal{U}$ for $K_n$. So, $\mathcal{C} = \bigcup_{n \in \mathbb{N}} \mathcal{F}_n$ is a countable subcover of $\mathcal{U}$ for $K$. Therefore $K$ is Lindel\"of.
	
	\end{proof}


	\begin{definition}
	  A subset \(K\) of a topological space \(X\) is said to be \textbf{hemicompact} if
	  there exist compact subspaces \(K_n\) of \(X\) for \(n\in\mathbb N\) such that
	  \(K=\bigcup_{n\in\mathbb N} K_n\), and for every compact subset \(H\subseteq K\),
	  \(H\subseteq K_n\) for some \(n\in\mathbb N\).
	\end{definition}

	\begin{theorem}
		Every hemicompact subset of a topological space is \(\sigma\)-compact.
	\end{theorem}
	\begin{proof}
		This proof is trivial as it is defined as stated here in definition 21.
	\end{proof}

	\begin{theorem}
		If $\mathcal{K}$ $\underset{pre}{\uparrow}$ $Gru_{K, P} (X)$ then $X$ is Lindel\"of.
	\end{theorem}
	\begin{proof}
	 It is shown in Theorem 2.5 (3, Clontz) that when $\mathcal{K}$ $\underset{pre}{\uparrow}$ $Gru_{K, P} (X)$, $X$ is hemicompact and furthermore by the definition of hemicompactness  $X = \lbrace \bigcup_{m \leq n} \sigma (m) : n < \omega \rbrace$.
	 \newline\newline
	 For $X = \lbrace \bigcup_{m \leq n} \sigma (m) : n < \omega \rbrace$, let $\mathcal{U}$ be an arbitrary covering of $X$ and let $\mathcal{F}_n = \lbrace U_{n0}, U_{n2}, ... , U_{nm}\rbrace$ be a finite subcover of $\mathcal{U}$ for $K_n$. So, $\mathcal{C} = \bigcup_{n \in \mathbb{N}} \mathcal{F}_n$ is a countable subcover of of $\mathcal{U}$ for $K$.\newline \newline
	 Alternate idea:
	 \newline
	 For $X = \lbrace \bigcup_{m \leq n} \sigma (m) : n < \omega \rbrace$, let $\mathcal{U}$ be an arbitrary covering of $X$ and let $\mathcal{F} \subseteq \mathcal{U}$, with $\mathcal{F} = \lbrace F_i : i < \omega \rbrace$. For each $\sigma(m)$ with $m \leq n < \omega$ let $F_{i < \omega} \supseteq \sigma(m)$. So, $\bigcup \mathcal{F} \supseteq \lbrace \bigcup_{m \leq n} \sigma(m) : n < \omega \rbrace$ so $X$ is Lindel\"of.
	\end{proof}
\end{document}
