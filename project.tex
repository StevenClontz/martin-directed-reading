\documentclass{article}

\usepackage{martin-style}

\begin{document}

	\begin{proposition}
	Puzzle 1: Find an infinite collection of open intervals in $\mathbb{R}$ whose intersection is not an open interval.
	\end{proposition}

	\begin{proof}
	$X = \bigcap\lbrace (\frac{-1}{n}, \frac{1}{n}) : n \in \mathbb{Z}^+\rbrace$
	\end{proof}
  \begin{proposition}
      Any finite union of closed sets is closed, and any arbitrary intersection of closed sets is closed.
  \end{proposition}

  \begin{proof}
  	Proposition 1 - Any finite union of closed sets is closed.
  	\newline
  	Let C, D be closed sets.
  	\newline
  	Let A, B be compliments of C, D so A, B are open.
  	\newline
  	Then, A $\cap$ B is also open.
  	\newline
  	Thus, X$\setminus$(A $\cap$ B) is closed.
  	\newline
  	X$\setminus$(A $\cap$ B) = X$\setminus$A $\cup$ X$\setminus$B = C $\cup$ D
  	\newline\newline
  	Inductive Proof:
  	\newline
  	Base Case: C $\cup$ D is closed.
  	\newline
  	Inductive Hypothesis:
  	\newline
  	Assume C$_1$ $\cup $ C$_2 \cup ...$ $\cup $ C$_n$ is closed for closed sets C$_i$.
  	\newline
	Then C$_1$ $\cup $ C$_2 \cup ...$ $\cup $ C$_n$ = K, K $\cup$ C$_{n+1}$ is closed.
  	\newline\newline
  	Proposition 1 - Any arbitrary intersection of closed sets is closed.
  	\newline
  	Let $\mathcal{C}$ be a collection of closed sets.
  	\newline
  	Let $\mathcal{U} = \lbrace$X$\setminus$C : C $\in \mathcal{C}\rbrace$, so $\mathcal{U}$ is a collection of open sets.
  	\newline
  	Then, $\bigcup \mathcal{U}$ is also open.
  	\newline
  	Thus, X$\setminus\bigcup\mathcal{U}$ is closed.
  	\newline
  	X$\setminus\bigcup\mathcal{U}$ = $\bigcap\mathcal{C}$
  	\newline
  	Therefore, $\bigcap\mathcal{C}$ is closed.
	\end{proof}

	\begin{lemma}
	A set U is open if and only if for every point $x \in$ U, there exists an open set U$_x$ where $x \in $ U$_x \subseteq$ U
	\end{lemma}
	\begin{proof} Suppose U is open
	\newline
		Then for all x $\in$ U there exists an open subset of U, U$_x$ such that x $\in$ U$_x$
	\end{proof}
	\begin{proof} Suppose there is an open set $\lbrace$U$_x$ : x $\in$ U$_x \rbrace \subseteq$ U.
	\newline
	We want to show that U is open.
	\newline
	If U$_x \subseteq$ U, then for all $x \in$ U$_x$, $x \in$ U
	\newline
	U$_x \subseteq$ U is open if U$_x \subseteq \tau$
	\newline
	If U$_x \subseteq \tau$, then $\bigcup$U$_x \in \tau$
	\newline
	$\bigcup$U$_x$ = U $\in \tau$, so U is open.
	\end{proof}
	\begin{proposition}
		A set K in a topological space X is closed if and only if K contains all its limit points.
	\end{proposition}
	\begin{proof} Suppose K contains all its limit points.
	\newline
	This is what you had already given
	\newline
	If K = X, then K is closed because $\emptyset$ is open.
	\newline
	Otherwise let x $\in$ X$\setminus$K, so x is not a limit point of K.
	\newline
	Then, $\exists$ $x \in$ U$_x \in \tau$ such that U$_x \cap$ K = $\emptyset$.
	\newline
	So, $x \in $ U$_x \subseteq$ X$\setminus$K.
	\newline
	Thus, $\bigcup \lbrace$U$_x$ : $x \in$ X$\setminus$K$\rbrace \supseteq$ X$\setminus$K and $\bigcup \lbrace$U$_x$ : $x \in$ X$\setminus$K$\rbrace \subseteq$ X$\setminus$K since U$_x \subseteq$ X$\setminus$K.
	\newline
	Therefore, $\bigcup \lbrace$U$_x$ : $x \in$ X$\setminus$K$\rbrace$ = X$\setminus$K, so X$\setminus$K is open and K is closed.
	\end{proof}
	\begin{proof} Suppose K is closed.
	\newline
		Still unsure how to apply the lemma to the double implication...also unsure if I have the lemma correctly in order to even use it in the first place...
	\end{proof}
	\begin{proposition}
	Verify the discrete and indiscrete topologies are topologies
	\end{proposition}
	\begin{proof} Prove the discrete topology is an actual topology.
	\newline\newline
	The discrete topology on a set X is $\tau$ = $\mathcal{P}($X$)$
	\newline
	$\emptyset$, X $\in \mathcal{P}($X$)$, So $\emptyset,$ X $\in \tau$
	\newline
	Let U $\in \tau$, or U $\subseteq \mathcal{P}($X$)$, then the $\bigcup$U $\in \mathcal{P}($X$)$ or $\bigcup$U $\in \tau$
	\newline
	Let U, V $\in \tau$, then the intersection of U, V is also in $\tau$.
	\end{proof}
	\begin{proof} Prove the indiscrete topology is an actual topology.
	\newline\newline
	The indiscrete topology on a set X is given as $\tau$ = $\lbrace \emptyset$, X$\rbrace$.
	\newline
	Clearly, $\emptyset$, X $\in \tau$.
	\newline
	Let $\mathcal{U}$ be a collection of all open sets in X, then the $\bigcup \mathcal{U}$ is also $\in \tau$.
	\newline
	Let U, V be open sets in X, then U, V $\in \tau$.
	\newline
	Thus the intersection f open sets U, V is also $\in \tau$
	\end{proof}
	\begin{proposition}
	Find a basis for the indiscrete topology on $\mathbb{R}$.
	\end{proposition}	
	\begin{proof}
	Find a basis for the indiscrete topology on $\mathbb{R}$.
	\newline\newline
	The indiscrete topology on $\mathbb{R}$ is given as $\tau$ = $\lbrace \emptyset$, $\mathbb{R}\rbrace$
	\newline
	A basis on $\mathbb{R}$ is a collection $\mathcal{B} \subseteq \mathcal{P}(\mathbb{R})$ s.t. for each $x \in \mathbb{R}$ $\exists$ B $\in \mathcal{B}$ : $x \in \mathcal{B}$.
	\newline
	Let $\mathcal{B}$ = $\mathbb{R}$. Since $\mathcal{B} \subseteq \mathcal{P}(\mathbb{R})$, for each $x \in \mathbb{R}$, there is a subset B $\in \mathcal{B}$ : $x \in \mathcal{B}$.
	\newline
	Thus, $x \in \mathcal{B}$ and B is a basis on $\mathbb{R}$.
	\newline
	Furthermore, if two sets B$_1$, B$_2$ $\in \mathcal{B}$, then their intersection is also in $\mathcal{B}$. Let $x \in$ B$_1$, B$_2$. Then $x \in$ B$_1 \cap$ B$_2$.
	\newline
	B$_1 \cap$ B$_2$ = B$_3$, so B$_3 \in \mathcal{B}$, and $x \in$ B$_3 \subseteq$ B$_1 \cap$ B$_2$.
	\newline
	Thus B$_3$ is also a basis on $\mathbb{R}$.
	\end{proof}
	
\end{document}
