\documentclass{amsart}
\usepackage{amsmath}
\usepackage{amsthm}
\usepackage{amssymb}



      \theoremstyle{plain}
      \newtheorem{theorem}{Theorem}
      \newtheorem{lemma}[theorem]{Lemma}
      \newtheorem{corollary}[theorem]{Corollary}
      \newtheorem{proposition}[theorem]{Proposition}
      \newtheorem{conjecture}[theorem]{Conjecture}
      \newtheorem{question}[theorem]{Question}
      \newtheorem{claim}[theorem]{Claim}

      \theoremstyle{definition}
      \newtheorem{definition}[theorem]{Definition}
      \newtheorem{notation}[theorem]{Notation}
      \newtheorem{example}[theorem]{Example}
      \newtheorem{game}[theorem]{Game}

      \theoremstyle{remark}
      \newtheorem{remark}[theorem]{Remark}





\begin{document}

\title{Relating games of Menger, countable fan tightness, and selective separability}


\author{Steven Clontz}
\address{Department of Mathematics and Statistics,
The University of South Alabama,
Mobile, AL 36688}
\email{sclontz@southalabama.edu}









\begin{abstract}
  By adapting techniques of Arhangel'skii, Barman, and Dow, we may
  equate the existence of perfect-information, Markov, and tactical
  strategies between two interesting selection games.
  These results shed some light on Gruenhage's question asking whether all
  strategically selectively separable spaces are Markov selectively
  separable.
\end{abstract}


\maketitle







\section{Introduction}

\begin{definition}
  Foo
\end{definition}

Bar.


\begin{question}\label{mainQuestion}
  Does \(SS^+\) imply \(SS^{++}\)?
\end{question}




\section{\(CFT\), \(CDFT\) and \(SS\)}

\begin{theorem}[Lemma 2.7 of ]
  The following are equivalent for any topological space \(X\).
  \begin{itemize}
    \item \(X\) is \(SS\).
    \item \(X\) is separable and \(CDFT\).
    \item \(X\) has a countable dense subset \(D\) where
          \(CDFT_x\) holds for all \(x\in D\).
  \end{itemize}
\end{theorem}



\end{document}
